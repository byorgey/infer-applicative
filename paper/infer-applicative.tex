% -*- mode: LaTeX; compile-command: "pdflatex infer-applicative.tex" -*-

\documentclass[sigplan,screen]{acmart}

%%
%% \BibTeX command to typeset BibTeX logo in the docs
\AtBeginDocument{%
  \providecommand\BibTeX{{%
    Bib\TeX}}}

%% Rights management information.  This information is sent to you
%% when you complete the rights form.  These commands have SAMPLE
%% values in them; it is your responsibility as an author to replace
%% the commands and values with those provided to you when you
%% complete the rights form.
\setcopyright{acmlicensed}
\copyrightyear{2024}
\acmYear{2024}
\acmDOI{XXXXXXX.XXXXXXX}

%% These commands are for a PROCEEDINGS abstract or paper.
\acmConference[ICFP '25]{The International Conference on Functional
  Programming}{September 10--12, 2025}{City, Country}
%%
%%  Uncomment \acmBooktitle if the title of the proceedings is different
%%  from ``Proceedings of ...''!
%%
%%\acmBooktitle{Woodstock '18: ACM Symposium on Neural Gaze Detection,
%%  June 03--05, 2018, Woodstock, NY}
\acmISBN{978-1-4503-XXXX-X/18/06}


%%
%% Submission ID.
%% Use this when submitting an article to a sponsored event. You'll
%% receive a unique submission ID from the organizers
%% of the event, and this ID should be used as the parameter to this command.
%%\acmSubmissionID{123-A56-BU3}

%%
%% The majority of ACM publications use numbered citations and
%% references.  The command \citestyle{authoryear} switches to the
%% "author year" style.
%%

\citestyle{acmauthoryear}

\newcommand{\term}[1]{\emph{#1}}

\begin{document}

%%
%% The "title" command has an optional parameter,
%% allowing the author to define a "short title" to be used in page headers.
\title{Idiomatic Inference}

\author{Brent A. Yorgey}
\email{yorgey@hendrix.edu}
\orcid{0009-0005-0135-6134}
\affiliation{%
  \institution{Hendrix College}
  \city{Conway}
  \state{Arkansas}
  \country{USA}
}

\begin{abstract}
  McBride and Paterson's \emph{idioms}, or \emph{applicative
    functors}, are a well-known abstraction defining function
  application in some ambient effectful context.  It is useful,
  especially in the context of designing domain-specific languages, to
  allow such ``idiomatic'' function application to be inferred,
  allowing the programmer to use plain function application syntax.
  We prove that this is possible in a strong sense, by presenting a
  formally-verified type inference algorithm that can automatically
  infer uses of an applicative functor and insert appropriate
  coercions.  We also prove that this cannot lead to any ambiguity: in
  cases where there are multiple valid type-correct ways to insert
  coercions, the applicative functor laws guarantee that there is no
  observable difference.  We also extend our results to Selective
  functors, but demonstrate via examples that Monad inference leads to
  ambiguity.
\end{abstract}

%%
%% The code below is generated by the tool at http://dl.acm.org/ccs.cfm.
%% Please copy and paste the code instead of the example below.
%%
\begin{CCSXML}
<ccs2012>
 <concept>
  <concept_id>00000000.0000000.0000000</concept_id>
  <concept_desc>Do Not Use This Code, Generate the Correct Terms for Your Paper</concept_desc>
  <concept_significance>500</concept_significance>
 </concept>
 <concept>
  <concept_id>00000000.00000000.00000000</concept_id>
  <concept_desc>Do Not Use This Code, Generate the Correct Terms for Your Paper</concept_desc>
  <concept_significance>300</concept_significance>
 </concept>
 <concept>
  <concept_id>00000000.00000000.00000000</concept_id>
  <concept_desc>Do Not Use This Code, Generate the Correct Terms for Your Paper</concept_desc>
  <concept_significance>100</concept_significance>
 </concept>
 <concept>
  <concept_id>00000000.00000000.00000000</concept_id>
  <concept_desc>Do Not Use This Code, Generate the Correct Terms for Your Paper</concept_desc>
  <concept_significance>100</concept_significance>
 </concept>
</ccs2012>
\end{CCSXML}

\ccsdesc[500]{Do Not Use This Code~Generate the Correct Terms for Your Paper}
\ccsdesc[300]{Do Not Use This Code~Generate the Correct Terms for Your Paper}
\ccsdesc{Do Not Use This Code~Generate the Correct Terms for Your Paper}
\ccsdesc[100]{Do Not Use This Code~Generate the Correct Terms for Your Paper}

%%
%% Keywords. The author(s) should pick words that accurately describe
%% the work being presented. Separate the keywords with commas.
\keywords{applicative, functor, inference, Agda}
%% A "teaser" image appears between the author and affiliation
%% information and the body of the document, and typically spans the
%% page.
% \begin{teaserfigure}
%   \includegraphics[width=\textwidth]{sampleteaser}
%   \caption{Seattle Mariners at Spring Training, 2010.}
%   \Description{Enjoying the baseball game from the third-base
%   seats. Ichiro Suzuki preparing to bat.}
%   \label{fig:teaser}
% \end{teaserfigure}

% \received{20 February 2007}
% \received[revised]{12 March 2009}
% \received[accepted]{5 June 2009}

\maketitle

\section{Introduction}

If we want to add two numbers, we can write an expression like
\[ |x + y|. \] But what if one of the numbers might not exist?  For
example, perhaps the number was parsed from user input, which might
not be in the right format.  We could represent the possibly-failing
number using some kind of option type, such as |Maybe| in Haskell, XXX
in other languages.  But now the simple expression |x + y| no longer
typechecks, since |x| is (say) of type |Int|, whereas |y| has the
incompatible type (say) |Maybe Int|.  Of course, we can pattern-match
on $y$, taking an appropriate action in each case.  In Haskell, we
might write this as
\begin{verbatim}
case y of
  Nothing -> Nothing
  Just n -> Just (x + n)
\end{verbatim}
but now the fact that we are essentially performing an addition
operation has been obscured by a lot of syntax.

In the Haskell community, \term{applicative functors}, first
introduced by McBride and Paterson XXX cite, are a well-known
solution to this situation.  An applicative functor is any type
constructor $\square$ which supports operations |pure| and |ap| having
the following types:

|pure : t -> A t|
|ap : A (s -> t) -> (A s -> A t)|

|pure| gives us a way to inject plain values of type |t| into the type
|A t|, and |ap| allows us to distribute |A| over arrows.  Together
with some laws XXX.



% The ``\verb|figure|'' environment should be used for figures. One or
% more images can be placed within a figure. If your figure contains
% third-party material, you must clearly identify it as such, as shown
% in the example below.
% \begin{figure}[h]
%   \centering
%   \includegraphics[width=\linewidth]{sample-franklin}
%   \caption{1907 Franklin Model D roadster. Photograph by Harris \&
%     Ewing, Inc. [Public domain], via Wikimedia
%     Commons. (\url{https://goo.gl/VLCRBB}).}
%   \Description{A woman and a girl in white dresses sit in an open car.}
% \end{figure}



\begin{acks}
Acknowledgements.
\end{acks}

%%
%% The next two lines define the bibliography style to be used, and
%% the bibliography file.
\bibliographystyle{ACM-Reference-Format}
\bibliography{references}

\appendix

\section{Proofs}

\end{document}
\endinput
%%
%% End of file `sample-sigplan.tex'.
